\section{Related Work}

Bouncy Castle is a FIPS-142 approved security library making it a good resource for security functionality, but while Bouncy Castle provides a plethora of security functions, it may be hard for developers to choose which specific functionalities and parameters to utilize in order to keep their product efficient and secure for an IoT environment. \cite{bc} Utilizing a security library correctly in itself is difficult, but in general, providing security for an IoT environment bears its own difficulties. 

The Bouncy Castle distribution for Java ME platforms is incredibly verbose, offering over 2000 java classes, over 340,000 lines of code, and support for dozens of algorithms for encryption, hashing, and more. Simply understanding the layout of Bouncy Castle's hierarchy may be daunting, or even cryptic, for users unfamiliar with security and cryptography. 

Understanding the taxonomy of Bouncy Castle's distribution is time consuming and can be complex. Users are required to know the proper order of function calls, necessary parameters, and security strengthening techniques, while also selecting which algorithms to utilize for an IoT environment. Bouncy Castle's taxonomy provides freedom but adds complexity. Many developers unfamiliar with Bouncy Castle (or cryptography in general) may want to provide security for their projects without learning all the necessary components. Our library provides much of this work for the user; it supplies functionality for key exchange, authentication, integrity, and encryption in {\bf eighteen} streamlined functions. Our methods provide the proper set up and execution of function chains required in order to provide easy to understand, ready-to-use security functionality that is quick and simple to integrate into a project. We cut a swath through the complexity and provide functions designed to mitigate time needed for users to add security functionality to their project by providing predetermined parameters such as proper elliptic curves, key sizes, and algorithms that provide effective security while remaining low in resource usage. Not only does our library reduce lines of code required by developers, time required to learn how to properly implement security methods, and time required to learn Bouncy Castle's taxonomy, but it also provides the most efficient methodology to bring security to an IoT environment. In the next section, we discuss the specific algorithms and schemes we utilized in order to achieve low resource intensive security for an IoT Smart Home.



