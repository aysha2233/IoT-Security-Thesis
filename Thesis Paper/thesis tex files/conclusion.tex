\section{Conclusion}
The introduction of smart devices into our homes may increase convenience, but security and privacy may also be effected. 
Smaller devices with fewer resources may not be capable of certain security mitigations,
and therefore, finding a balance between security and efficiency is a difficult task. 
%In this project, we investigated the use of Java Micro Edition and Bouncy Castle as a possible means to develop a security library for smart home devices.
In order to bring additional security to IoT environments,
we utilized Java Micro Edition and Bouncy Castle to create a customizable security library for key exchange, key generation, encryption, authentication, and integrity. 
Elliptic curve cryptography offers a faster, more efficient way to generate symmetric keys, %for use in a key exchange,
so we utilized ECDH to generate a shared secret and create a shared symmetric key even for devices that have few resources. 
Further, AES encryption with a relatively small key of 128 bits helps IoT environments encrypt quickly and communicate securely. 
%Symmetric cryptography is the backbone of secure communication in low resource systems like a smart home due to its incredible speed over asymmetric cryptography. 
Authentication and integrity are also important for well rounded security in an IoT environment,
but schemes like digital signatures which use asymmetric cryptography may be unsuitable for a low resource environment,
so we utilized MACs and authentication tags in our security library. 
%The use of MACs and authentication tags is useful in a low resource environment where we may not be able to spare the overhead to use schemes like digital signatures which use asymmetric cryptography. 
Our symmetric key MACs offer both the security and speed needed for authentication and integrity in an IoT environment.  
In our case study, we found that not only do these security functions run on IoT devices, but the overhead is even less than expected. 
While these functionalities add overhead to common operations,
this library may be a possible solution for certain IoT devices. 
Usage of this library may bring the currently lacking required security to many smart home environments while offering a customizable and easy to use interface. 
